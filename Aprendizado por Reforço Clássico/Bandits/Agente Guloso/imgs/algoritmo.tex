\documentclass[brazilian,preview]{standalone}
\usepackage[margin=0cm]{geometry}
\usepackage{babel}
\usepackage[babel, final]{microtype}
\usepackage{mathtools, amssymb, amsthm}
\usepackage[math-style=ISO]{unicode-math}
\usepackage[framemethod=TikZ]{mdframed}
\usepackage{listings}

\makeatletter
\lstnewenvironment{algorithm}{
    \lstset{
        columns=fullflexible,
        mathescape,escapechar=@,
        literate={-}{-}1,
        morekeywords={Parâmetro,Parâmetros,Inicializar,Loop},
    }}{}
\surroundwithmdframed[
    roundcorner=5pt,
    linewidth=2pt,
    linecolor=black!70,
    backgroundcolor=black!5,
    frametitlebackgroundcolor=black!70,
    frametitlefont={\normalfont\bfseries\color{white}},
    frametitle={Algoritmo\hspace*{.5em}\@title},
]{algorithm}
\makeatother


\title{Guloso}
\geometry{paperwidth=17cm}

\begin{document}
\begin{algorithm}
Parâmetros: $k$ braços
Inicializar de $a=1$ até k:
    $Q(a) \leftarrow 0$ @\hfill@ (Tabela dos valores esimados para cada ação $a$)
    $N(a) \leftarrow 0$ @\hfill@ (Tabela do número de ações para casa ação $a$)
Loop para cada episódio:
    $A \leftarrow argmax(Q)$ @\hfill@ (retorna o índice da ação com maior valor estimado em $Q$)
    $R \leftarrow bandit(A)$ @\hfill@ (puxa a alavanca $A$)
    $N(A) \leftarrow N(A) + 1$
    $Q(A) \leftarrow Q(A) + \frac{[R - Q(A)]}{N(A)}$
\end{algorithm}
\end{document}