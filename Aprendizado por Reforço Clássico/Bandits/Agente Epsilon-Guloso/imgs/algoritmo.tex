\documentclass[brazilian,preview]{standalone}
\usepackage[margin=0cm]{geometry}
\usepackage{babel}
\usepackage[babel, final]{microtype}
\usepackage{mathtools, amssymb, amsthm}
\usepackage[math-style=ISO]{unicode-math}
\usepackage[framemethod=TikZ]{mdframed}
\usepackage{listings}

\makeatletter
\lstnewenvironment{algorithm}{
    \lstset{
        columns=fullflexible,
        mathescape,escapechar=@,
        literate={-}{-}1,
        morekeywords={Parâmetro,Parâmetros,Inicializar,Loop, Se, Se não},
    }}{}
\surroundwithmdframed[
    roundcorner=5pt,
    linewidth=2pt,
    linecolor=black!70,
    backgroundcolor=black!5,
    frametitlebackgroundcolor=black!70,
    frametitlefont={\normalfont\bfseries\color{white}},
    frametitle={Algoritmo\hspace*{.5em}\@title},
]{algorithm}
\makeatother


\title{$\epsilon$-Guloso}
\geometry{paperwidth=17cm}

\begin{document}
\begin{algorithm}
Parâmetros: $k$ braços, $\varepsilon > 0$
Inicializar de $a=1$ até $k$:
    $Q(a) \leftarrow 0$ 
    $N(a) \leftarrow 0$ 
Loop para cada episódio:
    $x \leftarrow $ Um número escolhido aleatoriamente entre $[0, 1)$
    Se $x < \varepsilon$:
        $A \leftarrow$  Uma ação aleatória
    Se não: 
        $A \leftarrow \operatorname{argmax}(Q)$ 
    $R \leftarrow \operatorname{bandit}(A)$ @\hfill@ (puxa a alavanca $A$)
    $N(A) \leftarrow N(A) + 1$
    $Q(A) \leftarrow Q(A) + \dfrac{[R - Q(A)]}{N(A)}$

\end{algorithm}
\end{document}